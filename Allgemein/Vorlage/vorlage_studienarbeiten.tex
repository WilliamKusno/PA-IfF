% IfF LaTeX Vorlage f�r Projektberichte
% getestet mit MiKTeX 2.9
% von Jan H�fele
% 19.03.2012
% Weitergabe ohne Erlaubnis untersagt

% --- PRĄAMBEL ---

\documentclass[a4paper,DIV=18,BCOR=5mm,twoside=false,11pt,parskip=half,headsepline,headings=normal]{scrbook}
% KOMA-Klasse scrbook
% Seitenformat: DIN A4
% Satzspiegel: DIV 18
% Binderand: 5mm
% einseitig
% Schriftgr��e: 11pt
% halber Abstand zwischen Abs�tzen
% Trennlinie unter Kopfzeile

% Pakete
\usepackage[latin1]{inputenc} % Eingabe von Sonderzeichen
\usepackage[T1]{fontenc} % Moderner Schriftsatz
\usepackage[ngerman]{babel} % Anpassung des Dokuments auf neue, deutsche Rechtschreibung
\usepackage[osf,sc]{mathpazo} % Schriftart Mathpazo f�r den Mathematikmodus
\usepackage[plainheadsepline]{scrpage2} % scrpage2 Paket
\usepackage{pdfpages} % Einbinden mehrseitiger PDF Dokumente
\usepackage{graphicx} % Einbinden von Grafiken, bevorzugt als Vektorgrafik im PDF-Format
\usepackage[labelfont=bf]{caption} % fettgedruckte Captions
\usepackage{hyperref} % erzeugt Verweise im Dokument
\usepackage[all]{hypcap} % Hyperrefverweise f�hren immer an den Kopf der Verweisstellen
\usepackage[autostyle]{csquotes} % Zitieren mit \enquote{}
\usepackage[style=alphabetic,sorting=nyt,maxnames=2,maxbibnames=99]{biblatex} % Literaturverzeichnis mit dem biblatex Paket
\usepackage[intoc]{nomencl} % Formelverzeichnis mit dem nomencl Paket, ben�tigt makeindex
\usepackage[ngerman=ngerman-x-latest]{hyphsubst} % Hinzuf�gen eigener Worttrennungsregeln
\usepackage{flafter} % Floatumgebungen werden erst nach Erstellungspunkt angezeigt
\usepackage[morefloats=10]{morefloats} % erlaubt die Einbindung von mehr Floatumgebungen im PDF-Dokument
\usepackage[decimalsymbol=comma]{siunitx} % Werte mit Einheit k�nnen �ber den Befehl \SI{Wert}{Einheit} dargestellt werden
\usepackage{listings} % zum Einbinden von Quellcode
\usepackage{amssymb} % zus�tzliche Symbole
\usepackage{amsmath} % zus�tzliche Symbole
\usepackage{dsfont} % zus�tzliche Symbole
\usepackage{trfsigns} % zus�tzliche Symbole

% Schriftart und Zeilenabstand
\setkomafont{dictum}{\normalfont\normalcolor\rmfamily\small}
\renewcommand*\sectfont{\normalcolor\rmfamily\bfseries}
\renewcommand*\descfont{\rmfamily\bfseries}
\linespread{1.08}

% Definition eigener Worttrennungsregeln (bei Problemen mit �berh�ngenden Zeilen)
\hyphenation{} % W�rter hintereinander ohne Komma eingeben, Silbentrennung durch "-", z.B. "E-Ma-schi-ne"

% Definition mathematischer Funktionen, die standardm��ig nicht in LaTeX enthalten sind
\DeclareMathOperator{\sign}{sign}
\DeclareMathOperator{\imag}{Im}
\DeclareMathOperator{\real}{Re}

% Kopfzeile Einstellungen
\automark{chapter}
\pagestyle{scrheadings}
\renewcommand*{\chapterpagestyle}{plain}
\clearscrheadfoot
\ihead{\headmark}
\ohead[Seite \pagemark]{Seite \pagemark}

% Tabellen Einstellungen
\setlength{\tabcolsep}{5pt}
\renewcommand*{\arraystretch}{1.2}
\setlength{\itemsep}{0ex plus0.2ex}
\setlength{\parindent}{0ex}
\setlength{\belowcaptionskip}{0.4cm}
\setlength{\abovecaptionskip}{0.4cm}

% Formelverzeichnis Einstellungen, siehe ftp://ftp.fu-berlin.de/tex/CTAN/macros/latex/contrib/nomencl/nomencl.pdf
\renewcommand*{\nomname}{H�ufig verwendete Formelzeichen}
\setlength{\nomlabelwidth}{.2\hsize}
\renewcommand*{\nomlabel}[1]{#1 \hfill}
\setlength{\nomitemsep}{-\parsep}
\makenomenclature

% Literaturverzeichnis Einstellungen, siehe ftp://www.ctan.org/ctan/macros/latex/exptl/biblatex/doc/biblatex.pdf
\bibliography{bibliography}
\renewcommand*{\bibname}{Literaturverzeichnis}
\renewcommand*{\labelnamepunct}{\addcolon\addspace}
\renewcommand*{\labelalphaothers}{\textsuperscript{+}}
\renewcommand*{\mkbibnamelast}{\scshape}
\renewcommand*{\mkbibnamefirst}{\scshape}
\DefineBibliographyStrings{ngerman}{andothers={et\adddot\ al\adddot}}
\DefineBibliographyStrings{ngerman}{phdthesis={Dissertation}}

% Hyperref Einstellungen, siehe ftp://ftp.fu-berlin.de/tex/CTAN/macros/latex/contrib/hyperref/hyperref.pdf
\hypersetup{
pdftitle={Titel der Arbeit}, % Titel der Arbeit
pdfsubject={Art der Arbeit}, % Projektarbeit / Studienarbeit/ Diplomarbeit / Bachelorarbeit / Masterarbeit
pdfauthor={Name}, % Name
pdfkeywords={}, % Stichw�rter im Zusammenhang mit der Arbeit
pdfproducer={pdfLaTeX},
pdfborder={0 0 0},
bookmarksnumbered}

% Listings Einstellungen
\lstset{language=Matlab,numbers=left, numberstyle=\tiny, stepnumber=1, numbersep=5pt} % Voreinstellung f�r MATLAB-Code

% --- DOKUMENT ---

\begin{document}

\frontmatter

\pagenumbering{roman}
\setcounter{page}{1}

% Titelseite
\begin{titlepage}

\vspace*{-30mm} 
\hfill
\begin{minipage}{0.45\textwidth}
\includegraphics{fig_tubs.pdf} % Logo der TU Braunschweig
\end{minipage}
\hfill
\begin{minipage}{0.45\textwidth}
\includegraphics{fig_iff.pdf} % Logo des IfF
\end{minipage}

\vspace{60mm}
\begin{center}
{\huge{{\textsf{Titel\\}}}} % Titel der Arbeit

\vfill

%\includegraphics{fig_titel.pdf} % ggf. Titelbild
\end{center}

\begin{tabular}{l}
\Large Art der Arbeit
\end{tabular}

\vspace{5mm}

\begin{large}
\begin{tabular}{l}
Name\\ % Name
Technische Universit�t Braunschweig\\ % Uni
Matrikelnummer xxx % Matr. Nr.
\end{tabular}

\vspace{5mm}

\begin{tabular}{ll}
Ausgabedatum: & xxx\\ % Ausgabedatum der Aufgabenstellung eintragen
Abgabedatum: & xxx\\ % Abgabedatum der Arbeit eintragen
\end{tabular}

\vspace{5mm}

\begin{tabular}{ll}
Erstpr�fer: & Prof. Dr.-Ing. Ferit K���kay\\ % Erstpr�fer
Zweitpr�fer: & \\ % Zweitpr�fer
Betreuer: & % Betreuer
\end{tabular}
\end{large}

\end{titlepage}

% Leere Seite nach der Titelseite
\newpage
\thispagestyle{empty}
\null

% Aufgabenstellung, in das Verzeichnis dieser Datei kopieren
\newpage
%\includepdf[pages=1]{aufgabenstellung.pdf}

% Selbstst�ndigkeitserkl�rung
\addchap*{Selbstst�ndigkeitserkl�rung}
\noindent Ich versichere an Eides statt, dass ich diese Diplomarbeit selbstst�ndig verfasst und keine anderen als die angegebenen Quellen und Hilfsmittel verwendet habe. Alle Stellen, die w�rtlich oder sinngem�� aus anderen Quellen entnommen wurden, sind deutlich als solche gekennzeichnet. Diese Arbeit wurde in gleicher oder �hnlicher Form noch keiner anderen Pr�fungsbeh�rde vorgelegt.

\vspace{20mm}

\noindent Braunschweig, xxx % Datum eintragen und in der gedruckten Arbeit pers�nlich unterschreiben

% Inhaltsverzeichnis
\clearpage
\phantomsection
\pdfbookmark[1]{Inhaltsverzeichnis}{toc}
\tableofcontents

% Abbildungsverzeichnis
\newpage
\phantomsection
\addcontentsline{toc}{chapter}{Abbildungsverzeichnis}
\listoffigures

% Tabellenverzeichnis
\newpage
\phantomsection
\addcontentsline{toc}{chapter}{Tabellenverzeichnis}
\listoftables

% Formelverzeichnis
\newpage
\markboth{\nomname}{\nomname}
\printnomenclature

\mainmatter

\addchap{Kurzfassung}

% Kurze Inhaltsangabe der Arbeit (etwa 10 Zeilen)

\chapter{Einleitung}

\chapter{Kapitel 1}

\section{Abschnitt 1}

\subsection{Unterabschnitt 1}

\section{Abschnitt 2}

\chapter{Kapitel 2}
 
\chapter{Zusammenfassung und Ausblick}

% Literaturverzeichnis

\newpage
\phantomsection
\addcontentsline{toc}{chapter}{Literaturverzeichnis}
\printbibliography[{title=Literaturverzeichnis}]

\appendix

\chapter{Anhang}

\end{document}